% !TEX root = ../main.tex
\section{模型的评价与改进}
	\subsection{优点}
		\begin{enumerate}
			\item 针对调度模型,相比于直接求取全局最优解的算法,先利用静态调度算法获得局部最优解集合,再利用模拟退火算法对局部最优解集合进行全局优化获得最优解,这样的模式不仅减小了运算量,而且降低了算法难度,同时也符合实际复杂的工程背景。
			\item 针对分配与分布模型,相比于暴力搜索所有可能性或者直接依据优先级确定结果,利用可行解规则减小可行解数量、确定质量较高的初始解,再利用禁忌算法迭代最优解,既保证了解的正确性、也保证了运算量有限性。
		\end{enumerate}
	\subsection{缺点}
		\begin{enumerate}
			\item 对于模拟退火算法和禁忌算法这类启发式优化算法,算法有效性很大程度上取决于参数设定,因此针对复杂程度不同的生产系统,具体参数的设定有所不同。
		\end{enumerate}
	\subsection{改进}
		对于复杂程度不同的生产系统区别对待:
		\begin{enumerate}
			\item 较为简单的生产系统可以考虑将模拟退火算法、禁忌算法终止条件放宽、增加迭代次数寻求全局最优解;
			\item 较为复杂的生产系统可以考虑将模拟退火算法、禁忌算法终止条件更严格、减小运算量,以相对较快逼近最优解为目的。
		\end{enumerate}
	\subsection{推广}
		本模型可以推广至更为复杂的多轨分布式车间生产制作环境。
		\par\indent 在实际生产过程中,一般为多轨、多自动引导车、多数控机床的复杂生产环境,此时求得自动引导任务调动的最优解运算量过大、运算负担过大,不符合生产要求,因此选用本文以运算量较小的局部最优解为基础再进行全局优化的思路能够大大减小运算量。
		\par\indent 在多轨分布式系统中,还需要考虑到最短路径规划、引导车防撞约束等限制条件,综合考虑RGV利用率和CNC平均等待时间确定任务调度系统。
