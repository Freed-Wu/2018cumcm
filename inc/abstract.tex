% !TEX root = ../main.tex
\setcounter{page}{1}
\maketitle
\begin{abstract}
	近年来随着制造业生产及物流业运输等产业的快速发展,由计算机数控机床(Computer Number Controller,CNC)及轨道式智能引导车(Rail Guide Vehicle,RGV)等构成的智能加工系统逐渐成为主流,大大提高了生产效率,而复杂生产系统的调度模型能够进一步提高生产效率,因此研究调度系统具有现实意义。
	\par\indent 考虑到生产中的调度与操作系统进程处理的相似性,我们将任务调度类比于操作系统,RGV相当于CPU,CNC需求信号相当于等待处理的进程,因此根据操作系统概念定义了RGV工作周期、CNC生产周期,而每次迭代过程为一个机器周期,易于理解。\index{key}
	\par\indent 针对任务一,建立调度一般解法,将其分为三个部分:一是任务调度模型;二是确定多工序时CNC分配与分布模型;三是故障模拟模型。
	\par\indent 其中任务调度模型分为两个步骤:一是求取局部最优解,二是以局部最优解作为初始解,利用启发式算法进行全局优化。相比于直接求取最优解,这种分步式模型既减小了运算量、降低了实现难度,也保证了解的优质性。针对求取局部最优解以及全局优化存在的多种算法,通过综合比较解质量、运算量等参数,我们最终选用电梯扫描算法求取局部最优解,模拟退火算法进行全局优化;其中多工序时CNC分配与分布模型也分为两个步骤:一是设立可行解条件确定可行解集合以及当前最优解,二是通过禁忌算法以当前最优解为初始解,进行全局逐步寻优确定最终解;其中故障模拟模型也分为两步:一是针对服从独立且参数相同的两点分布的故障发生的模拟,随机生成一个在1到100之间的随机整数,若该数等于1,则认为发生故障;二是针对服从独立且参数相同的均匀分布的故障发生时间的模拟,随机生成一个在0到1之间的随机数,该数乘以加工时间即为故障发生时间。
	\par\indent 针对任务二,首先建立评价指标,包括成料数量、总计CNC等待时间等。然后对于一道工序有无故障、两道工序有无故障四种情况,利用题目中的三组数据分别验证任务一中调度一般解法的有效性。
	\par\indent 其中对于任务调度模型,我们发现一道工序系统中模拟退火最优解与电梯扫描解相同,考虑到一道工序系统过于简单,再次比较两道工序系统中的结果,发现模拟退火算法成料多于电梯扫描算法,证明复杂系统下利用模拟退火进行全局寻优算法是有效的;其中对于CNC分配与分布模型,比较三组不同数据下CNC分布与分配,发现其明显不同,且改变分布分配后解质量下降,证明了禁忌搜索算法的有效性。综上所述,我们任务一中建立的模型及一般解法具有通用性,且同时保证了解的优越性与运算量有限性。
	\par\textbf{关键词:}RGV;操作系统;任务调度;故障模拟;启发式算法。
\end{abstract}
%\clearpage
%\begin{center}
%	\normalsize\textbf{Summary}
%\end{center}
%\par\small
%\begin{spacing}{0.5}
%	\\\textbf{Keywords: }
%\end{spacing}
%\normalsize
