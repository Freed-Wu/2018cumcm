% !TEX root = ../main.tex
\section{总结}
	针对调度模型,考虑到故障的随机性,因此我们选用无故障时运行结果进行比较。表\ref{一道工序算法比较}以第一组数据一道工序为例,由此可以看出,对于求取局部最优解的三种算法,电梯扫描算法解的质量最好,虽然时间相对较多,但考虑到初始解的质量很大程度上决定了模拟退火算法的效率和效果,且三种算法平均运算时间相差不大,因此我们选择电梯扫描算法求取局部最优解;将电梯扫描算法的解作为初始解,利用模拟退火全局寻优,发现对于此种情况模拟退火最优解并没有更加逼近上界 \footnote{gap\%=(上界-最优解)/最优解\(\times\)100\%。}。考虑到一道工序系统相对简单,因此我们再次对比电梯扫描算法、模拟退火算法在两道工序系统下的结果,由表\ref{两道工序算法比较}发现模拟退火算法247个成料优于电梯扫描算法245个成料,因此在复杂系统下利用模拟退火进行全局寻优算法是有效的。
	\par\indent 针对CNC分配与分布模型,首先由表\ref{CNC工序分配分布},可以看出针对三组不同的数据CNC分布与分配明显不同,证明禁忌搜索算法是有效的。其次为了验证解的优越性,我们以第一组数据、两道工序、无故障为例,发现改变CNC分布之后,新解小于现有解,证明解质量很高。
\begin{table}[htbp]
	\centering
	\caption{CNC工序分配分布}
	\label{CNC工序分配分布}
		\begin{longtabu}to\linewidth{@{}X[4,c]|*8{X[c]}@{}}
			\toprule
			\diagbox{组号}{结果}{CNC序号} & 1 & 2 & 3 & 4 & 5 & 6 & 7 & 8 \\ \midrule
			1  & 1 & 2 & 1 & 2 & 1 & 2 & 2 & 1 \\
			2  & 1 & 2 & 2 & 2 & 2 & 1 & 2 & 1 \\
			3  & 1 & 2 & 2 & 1 & 2 & 1 & 1 & 1 \\ \bottomrule
		\end{longtabu}
\end{table}
\begin{table}[htbp]
	\centering
	\caption{一道工序算法比较}
	\label{一道工序算法比较}
	\begin{longtabu}to\linewidth{@{}X[1.5,c]|*5{X[c]}@{}}
		\toprule
		\diagbox{算法}{结果}{指标}  & 最优解 & 平均结果     & 平均时间     & 上界  & gap\% \\ \midrule
		先到先服务  & 356 &    \textbackslash{}      & 0.006612 & 376 & 5.30 \\
		最短路径优先 & 356   & \textbackslash{} & 0.005288     &  376   & 5.30      \\
		电梯扫描   & 369 &      \textbackslash{}    & 0.007914 & 376 & 1.86\\ 
		模拟退火   & 369 &     366.45    & 2.509055 & 376 & 1.86\\ \bottomrule
	\end{longtabu}
\end{table}
\begin{table}[htbp]
	\centering
	\caption{两道工序算法比较}
	\label{两道工序算法比较}
	\begin{longtabu}to\linewidth{@{}X[1.5,c]|*5{X[c]}@{}}
		\toprule
		\diagbox{算法}{结果}{指标}  & 最优解 & 平均结果             & 平均时间     & 上界  & gap\% \\ \midrule
		先到先服务  & 235 & \textbackslash{} & 0.00591  & 263 & 10.6    \\
		最远寻道优先 & 235 & \textbackslash{} & 0.00726  & 263 &  10.6    \\
		电梯扫描   & 245 & \textbackslash{} & 0.00946  & 263 &  6.8    \\ 
		模拟退火   & 247 & 243.8762         & 2.393856 & 263 &   6.08   \\ \bottomrule
	\end{longtabu}
\end{table}
\begin{table}[htbp]
	\centering
	\caption{无故障时算法比较}
	\label{无故障时算法比较}
	\begin{longtabu}to\linewidth{@{}X[c]|*3{X[c]}@{}}
		\toprule
		\diagbox{指标}{结果}{算法} & 电梯算法  & 模拟退火算法 & 先来先服务  \\ \midrule
		个数 & 246   & 246    & 235   \\
		时间 & 28760 & 28788  & 28780 \\ \bottomrule
	\end{longtabu}
\end{table}
\begin{table}[htbp]
	\centering
	\caption{有故障时算法比较(3组平均)}
	\label{有故障时算法比较(3组平均)}
	\begin{longtabu}to\linewidth{@{}X[c]|*3{X[c]}@{}}
	\toprule
	\diagbox{指标}{结果}{算法} 	& 电梯算法     & 模拟退火算法      & 先来先服务     \\ \midrule
	个数 & 212.6667 & 210.3333333 & 205.6667 \\
	时间 & 28775.33 & 28628.66667 & 29121.33 \\ \bottomrule
	\end{longtabu}
\end{table}

